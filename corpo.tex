% ------------------------------------------------------------------------
% TCC1..2: Modelo de Documento para TCC e entrega para Banca - SENAI
% 
% Template por Andreza Quintas
%		(https://orcid.org/0000-0002-8169-6884)
%		(https://lattes.cnpq.br/8803951686659455)
% 24.mar versao 1
% ------------------------------------------------------------------------
% Agradecimentos a Overleaf pela oportunidade 
% ------------------------------------------------------------------------

%!TEX root = main.tex

\thispagestyle{empty}

\begin{center}
    \textblockorigin{-18pt}{-2pt}
    \begin{textblock*}{10cm}(3cm,1cm)
    \centering
    \includegraphics[width=16cm]{Logo SESI-SENAI 2.png}
    \end{textblock*} ~\\[2em]
\end{center}

\begin{center}
%verificar o titulo, pode melhorar....
{\bf \large \MakeUppercase{Titulo do Projeto Aqui}}\\[2em]
\end{center}


Acadêmico: \textbf{\MakeUppercase{Nome Completo do Aluno Aqui}} 

Semestre Letivo: \textbf{2025/1} 

Prof. Orientador \textbf{\MakeUppercase{Andreza Quintas}} 

Prof. Coorientador \textbf{\MakeUppercase{Se Houver}} 
\justify \linespread{1.5}


\section{O Problema a ser Pesquisado}

Parágrafo 1cm, texto justificado, espaçamento entre linhas 1,5, fonte Times New Roman ou Arial (Caso usar Arial modificar a fonte dos títulos e subtítulos), remover espaço antes ou depois do parágrafo (se houver), tamanho da fonte 12. Palavras americanas em Itálico, caso for necessário dar ênfase em alguma palavra deixar a mesma em negrito. 

Conforme \citeonline[p.~1]{citacao1} "Isso é um exemplo de citação indireta com menos de 3 linhas".
Isso é um exemplo de citação direta, com menos de 3 linhas. \cite{citacao2}.

\begin{citedireto}
[\ldots]Aqui é um exemplo de citação com mais de 3 linhas, em que é necessário realizar o recuo no parágrafo e fica separado de todo o corpo do texto. Aqui é um exemplo de citação com mais de 3 linhas, em que é necessário realizar o recuo no parágrafo e fica separado de todo o corpo do texto. Aqui é um exemplo de citação com mais de 3 linhas, em que é necessário realizar o recuo no parágrafo e fica separado de todo o corpo do texto. \cite[p.~324]{citacao3}.
\end{citedireto}

% Isso é um comentário, tudo que for escrito aqui não aparece no documento oficial.

Aqui deve continuar o texto, nunca se finaliza um texto com uma citação, deve ter um comentário ou um argumento em seguida.

\section{Objetivos}
%Aqui estão definidos os objetivos deste projeto, serão divididos em objetivo geral, o qual está amplamente apresentado o propósito deste projeto e os objetivos específicos apresentará a relação de objetivos que deverão ser seguidos para alcançar o objetivo geral. 

%O objetivo de um projeto é o resultado que se pretende alcançar, sendo um ponto de referência para a sua execução. 

\subsection{Objetivo Geral}

O objetivo geral de um projeto é a finalidade principal do trabalho, que resume a ideia central e delimita o escopo do projeto. 

\subsection{Objetivos específicos}
Os objetivos específicos são:
%O objetivo específico de um projeto é o resultado concreto que se pretende alcançar para atingir o objetivo geral. São os passos necessários para alcançar o objetivo geral. 
\begin{itemize}
	\item São frases completas, mas curtas;
        \item Devem ser mensuráveis, específicos e comprováveis;
	\item Devem expressar uma só ação por objetivo;
	\item Devem ser redigidos utilizando verbos no infinitivo.
\end{itemize}

\section{Justificativa}

A justificativa de um projeto é uma explicação escrita que demonstra a necessidade de realizar uma determinada iniciativa. É um componente essencial do projeto, que deve ser bem elaborado. 

\section{Referencial Teórico}

O referencial teórico é uma revisão de pesquisas e discussões de outros autores sobre o tema de um projeto de pesquisa. Ele é também chamado de referencial bibliográfico ou fundamentação teórica. 
O referencial teórico é um elemento textual que serve como base para o desenvolvimento do tema. Ele garante a qualidade científica do trabalho. 
Objetivos do referencial teórico

Fornecer um esquema completo para desenvolver a pesquisa 

Demonstrar que o autor fez uma revisão bibliográfica de maneira correta 

Mostrar que o autor está inteirado sobre o que de mais atual está sendo pesquisado dentro do tema escolhido 

Se for necessário colocar notas de roda pé para indicar o significado sem colocar dentro do corpo do texto \textit{Exemplo\footnote{Exemplo}}  e \textit{Exemplo2 \footnote{Exemplo 2}}, posso adicionar quantos forem necessários no documento e onde for necessário.


\section{Recursos}
Este capítulo descreve os recursos necessários para a realização do projeto. O objetivo é apresentar com clareza como e com o quê o projeto será desenvolvido, garantindo organização, planejamento e viabilidade.

O capítulo está dividido em três seções: Recursos Humanos, Ferramentas e Custos.

\subsection{Recursos Humanos}

Refere-se à equipe envolvida diretamente na execução do projeto. Neste caso, considera-se a participação de:

\begin{itemize}
\item Desenvolvedor: responsável... \textit{backend} %utiliza-se quando necessário colocar a palavra americana em itálico;
\item Função 2...: \textit{Item} % Somente utiliza-se itálico quando a palavra for americana ou de qualquer outro idioma diferente do português.
\end{itemize}

\subsection{Ferramentas e Tecnologias}
São os softwares, plataformas e ambientes utilizados na criação do projeto:

\subsubsection {Figma} Inserir a logomarca da ferramenta e descrever suas características técnicas, abordando sua finalidade, justificativa para adoção no contexto do projeto e o impacto esperado na eficiência, organização ou qualidade do produto final.

% Criar quantas subsection for necessário para falar sobre as ferramentas.


\subsection{Custos}
Embora este seja um projeto acadêmico, é importante estimar os custos envolvidos, principalmente em um cenário real. Os principais pontos de custo incluem:

Infraestrutura...


\section{Análise de Riscos}
  
A análise de risco de um projeto é um processo que identifica, avalia e gerencia os riscos que podem afetar o projeto.

\section{Métodos}
  
Um método de projeto é uma técnica de execução que faz parte de uma metodologia de projeto. A metodologia é um conjunto de estratégias, técnicas, processos e práticas que orientam o planejamento e a execução de um projeto. 

\section{Desenvolvimento}

Nessa section estará todo o desenvolvimento do seu projeto, processo de criação das telas no figma, processo de desenvolvimento HTML, CSS, JS, bem como os prints das telas e códigos principais e necessários.

\section{Cronograma}

Logo abaixo pode se ver o cronograma das atividades a serem realizadas em 2025/1 ou 2:% Agora será o 1 e semestre que vem o 2.

\begin{table}[!h]
\centering
\caption{Cronograma 2025/1}
\label{table:1}
\begin{tabular}{|l|l|l|l|l|l|l|}
\hline
\multicolumn{1}{|c|}{\textbf{Etapas}}            & Fev                      & \multicolumn{1}{c|}{Mar} & Abr                      & Mai                      & Jun\\ \hline
Tarefa 1                   & \cellcolor[HTML]{9B9B9B} & \cellcolor[HTML]{9B9B9B} &                          &                          &          \\ \hline
Tarefa 2       &                          & \cellcolor[HTML]{9B9B9B} & \cellcolor[HTML]{9B9B9B} & \cellcolor[HTML]{9B9B9B} &                \\ \hline
Tarefa 3                   &                          & \cellcolor[HTML]{9B9B9B} & \cellcolor[HTML]{9B9B9B} & \cellcolor[HTML]{9B9B9B} &                  \\ \hline
Tarefa 4                  &                          & \cellcolor[HTML]{9B9B9B} & \cellcolor[HTML]{9B9B9B} &                          &                 \\ \hline
Tarefa 5                   &                          &                          & \cellcolor[HTML]{9B9B9B} & \cellcolor[HTML]{9B9B9B} &             \\ \hline
Tarefa 6...                     &                          &                          &                          & \cellcolor[HTML]{9B9B9B} & \cellcolor[HTML]{9B9B9B} \\ \hline
Finalização - entrega final                     &                          &                          &                          &     & \cellcolor[HTML]{9B9B9B} \\ \hline
\end{tabular}
\\[0.5em] {Fonte: O autor.}
\end{table}
 
 \pagebreak
 